\documentclass[fleqn,usenatbib]{mnras}

\usepackage{newtxtext,newtxmath}
\usepackage[T1]{fontenc} % modern encoding

\usepackage{graphicx}	% Including figure files
\usepackage{amsmath}	% Advanced maths commands
%\usepackage{amssymb}	% Extra maths symbols

%% --- user packages ---
\usepackage{csquotes}
\usepackage{bm}
\usepackage{subcaption}
\usepackage{booktabs}
\usepackage{grffile}
\usepackage{overpic}
\usepackage{float}
\usepackage{enumitem}
\usepackage{amsfonts}
\usepackage{easy-todo}
\usepackage{listings}
%\usepackage[citecolor=blue]{hyperref}
\usepackage{lineno}

% Package options

% Url style
\hypersetup{
    colorlinks=true,
    linkcolor=blue,
    filecolor=magenta,      
    urlcolor=cyan,
}
 
\urlstyle{same}

\newdimen\figrasterwd
\figrasterwd\textwidth

%%%%% AUTHORS - PLACE YOUR OWN COMMANDS HERE %%%%%

\newcommand{\rs}[2]{{#2}}
\newcommand{\jq}[2]{{#2}}

% Make tensors and vectors bold
\newcommand{\ten}[1]{{\bm #1}}
\renewcommand{\vec}[1]{{\bm #1}}
\newcommand{\norm}[1]{\left\lVert#1\right\rVert}
\newcommand{\mycaption}[2]{\caption[#1]{\emph{#1} #2}}

%% --- end user commands ---

\title{Flute and kink instabilities in a dynamically twisted
    flux tube with anisotropic plasma viscosity}

\author[J.~Quinn and R.~Simitev]{
James J.~Quinn,$^{1}$\thanks{E-mail: jamiejquinn@jamiejquinn.com}
Radostin D.~Simitev$^{2}$
\\
% List of institutions
$^{1}$ Research Software Development Group, University College London, Gower Street, London WC1E 6BT, UK \\
\href{https://orcid.org/0000-0002-0268-7032}{orcid.org/0000-0002-0268-7032} \\
$^2$ School of Mathematics and Statistics, University of Glasgow,
Glasgow G12 8QQ,
UK \\ \href{https://orcid.org/0000-0002-2207-5789}{orcid.org/0000-0002-2207-5789}\\
}

\date{Accepted XXX. Received YYY; in original form ZZZ}

\pubyear{2021}

\begin{document}
\label{firstpage}
\pagerange{\pageref{firstpage}--\pageref{lastpage}}
\maketitle

\lstset{
  language=[90]Fortran,
  basicstyle=\ttfamily,
  keywordstyle=\color{red},
  commentstyle=\color{green},
  frame=single
}

\graphicspath{{images/kink_instability_straight/}}

\begin{abstract}
Magnetic flux tubes such as those in the solar corona are subject to a
number of instabilities. Important among them is the kink instability
which plays a central part in the nanoflare theory of coronal heating,
and for this reason in numerical simulations it is usually induced by
tightly-controlled perturbations and studied in isolation. In
contrast, we find that when disturbances are introduced in our
magnetohydrodynamic flux tube simulations by dynamic twisting of the
flow at the boundaries fluting modes of instability are readily
excited. We also find that the flute instability, which has been
theorised but rarely observed in the coronal context,  is strongly
enhanced when plasma viscosity is assumed anisotropic.
We proceed to investigate the co-existence and competition between
flute and kink instabilities for a range of values of the resistivity
and of the parameters of the anisotropic and isotropic models of viscosity.
We conclude that while the flute instability cannot prevent the kink
from ultimately dominating, it can significantly delay its development
especially at strong viscous anisotropy induced by intense magnetic fields.
\end{abstract}

\begin{keywords}
Sun: corona -- Sun: magnetic fields -- instabilities -- plasmas -- MHD
\end{keywords}

\section{Introduction}

The helical kink instability is a form of ideal magnetohydrodynamic
(MHD) instability which occurs in highly twisted magnetic flux tubes
such as those making up much of the solar
corona~\citep{realeCoronalLoopsObservations2014} and has been well
studied in the coronal
context~\citep{hoodKinkInstabilitySolar1979,hoodCoronalHeatingMagnetic2009,browningSolarCoronalHeating2003b,barefordShockHeatingNumerical2015,quinnEffectAnisotropicViscosity2020}. Given
its energetic nonlinear development, it is considered a potential
mechanism for heating the solar corona through the theory of
nanoflares~\citep{klimchukSolvingCoronalHeating2006,browningMechanismsSolarCoronal1991}
and a key mechanism in the production of solar
flares~\citep{hoodKinkInstabilitySolar1979}. Some of our previous work has investigated a twisted magnetic flux tube already linearly
unstable to the helical kink instability, focussing specifically on
the effect of anisotropic viscosity on the nonlinear
dynamics~\citep{quinnEffectAnisotropicViscosity2020}. In~\citet{quinnEffectAnisotropicViscosity2020}
and most other investigations of the kink instability
e.g.~\citet{hoodCoronalHeatingMagnetic2009}, a perturbation 
is applied to an already significantly twisted flux tube. An
alternative way to excite the instability (and the way employed here)
is to start with an initially straight field and apply twisting
motions at the boundaries to form a twisted flux tube which eventually
becomes unstable. This kind of dynamic excitation of the kink
instability is useful in that it represents more closely actual evolution of magnetic flux tubes and the associated instabilities in 
the solar corona. In our simulations, the
dynamic twisting of the flux tube reveals an additional instability,
the flute instability, which has been theorised (for example,
in~\citet{priestMagnetohydrodynamicsSun2013}). While oscillations resembling flute perturbations have been found in simulations of coronal loops~\citep{terradasEffectMagneticTwist2018}, to our knowledge, this is the first time the flute instability has been investigated computationally in a coronal context.

The flute instability arises in magnetised plasmas where
the plasma pressure gradient is oriented in the same direction as the
field line curvature, that is the pressure and magnetic tension forces
compete. This is similar to the competition between pressure and
gravitational forces which gives rise to the Rayleigh-Taylor
instability (RTI). In magnetohydrodynamic terminology, the RTI is a typical example of an ideal
interchange instability, where magnetic field lines are minimally bent
and are, instead, exchanged during the evolution of the
instability. The ideal flute instability is another example of an
ideal interchange instability but confined to a cylindrical geometry,
the term ``flute instability'' referring to its likeness to a fluted
column. In a twisted flux tube like a simple, unbraided coronal loop,
the magnetic curvature is always directed towards the axis so the tube
may be unstable to fluting when the pressure decreases outwards from a
high-pressure core. Such a pressure distribution is generated in the
flux tubes studied here as a result of the driving. The appearance of
the flute instability is illustrated by, for example, the pressure
contours in figure~\ref{fig:kink_field_line_plots}, where the
perturbations follow the pitch of the twisted field. 

In other solar contexts, interchange instabilities can be found in the
form of ballooning modes in
arcades~\citep{hoodBallooningInstabilitiesSolar1986}, as the
instability which forms tubes of specific size in the
photosphere~\citep{bunteInterchangeInstabilitySolar1993} and in the
buoyancy of flux
tubes~\citep{schuesslerInterchangeInstabilitySmall1984}. However, the
flute instability specifically is more commonly studied in fusion
contexts~\citep{mikhailovskiiInstabilitiesConfinedPlasma1998,zhengAdvancedTokamakStability2015,wessonHydromagneticStabilityTokamaks1978}. In
fusion, the focus is generally on understanding how a particular
plasma device may be stabilised to the instability in particular
geometries such as that of the mirror
machine~\citep{jungwirthTheoryFluteInstability1965} or in toroidal
geometries such as the
tokomak~\citep{shafranovFluteInstabilityCurrentcarrying1968}. The
resistive flute instability (also known as the resistive interchange
instability) can be excited even when the ideal flute instability is
stabilised. As a result, this has been given significantly more
attention~\citet{johnsonResistiveInterchangesNegativeV1967,correa-restrepoResistiveBallooningModes1983}. While
this body of research is useful and applicable in solar contexts, it
is mostly limited to the study of the stability and linear development
of the flute instability, the nonlinear development being of
secondary importance in the investigation of fusion devices. More
detailed investigations of its nonlinear development is required to
understand its importance in the context of coronal dynamics and
coronal heating. The development of the flute instability and its
interaction with the simultaneously growing kink instability is the
main focus of this work and the experiments described here represent an
initial exploration into the nonlinear flute instability in the
solar corona.

In addition to our main goal, of particular interest here is
also the effect of anisotropic plasma viscosity, which in the following is found to
strongly influence the growth of the flute instability.
It is well known that viscosity in magnetised plasmas (such as
those which make up the solar corona) is
anisotropic and strongly dependent on the strength and direction of
the local magnetic
field~\citep{hollwegViscosityChewGoldbergerLowEquations1986,hollwegViscosityMagnetizedPlasma1985,braginskiiTransportProcessesPlasma1965}.
To take this into account, MacTaggaret et al. developed a phenomenological model of anisotropic viscosity~\citep{mactaggartBraginskiiMagnetohydrodynamicsArbitrary2017} that
captures the main physics of viscosity in the 
solar corona as outlined in the analysis of
\cite{braginskiiTransportProcessesPlasma1965}, namely parallel
viscosity in regions of strong field  strength and isotropic viscosity
in regions of very weak or zero field strength. For brevity, we will
refer to this model of viscosity as ``the switching model''. In~\citet{quinnEffectAnisotropicViscosity2020,quinnKelvinHelmholtzInstabilityCollapse2021} we implemented
the switching model as a module for the widely-used general MHD code
Lare3d~\citep{arberStaggeredGridLagrangian2001}, and demonstrated
significant effects of anisotropic viscosity on the 
development of the nonlinear MHD kink instability and the
Kelvin-Helmholtz instability. More generally, the interest in
anisotropic viscosity stems from the open question of
which heating mechanism (viscous or Ohmic) is dominant in the solar
corona~\citep{klimchukSolvingCoronalHeating2006}, an important facet
of solving the coronal heating problem. Using scaling laws, it has
been suggested that viscous heating (generated through anisotropic
viscosity) can dwarf that of Ohmic
heating~\citep{craigAnisotropicViscousDissipation2009a,litvinenkoViscousEnergyDissipation2005}.
However,
due to computational and observational limitations, this cannot be
directly tested, and so the influence of other factors such as small
scale instabilities and turbulence is relatively
unknown~\citet{klimchukSolvingCoronalHeating2006}. In addition to
directly heating the plasma, viscosity plays a part in the damping of
instabilities and waves~\citep{rudermanSlowSurfaceWave2000}. It is
this effect we are most interested in here, and it shall be reported that
the use of anisotropic viscosity permits the growth of the
flute instability, which is otherwise strongly damped by
isotropic viscosity. 

This paper is organised as follows. Section~\ref{sec-flute-intro} introduces the flute instability and relevant linear analyses, section~\ref{sec-numerical-setup} describes the governing equations, coronal loop model and details of the numerical parameters, section~\ref{sec-results} presents the overall development of the flute instability, before discussing features unique to the two values of resistivity studied, section~\ref{sec-discussion} discusses the limitations of the simulations, with suggestions for future work, and section~\ref{sec-conclusions} presents our conclusions in the wider context of coronal heating.

\subsection{The flute instability}
\label{sec-flute-intro}

In general, the stability of a cylindrical twisted magnetic flux tube is analysed using perturbations of the form
\begin{equation}
  \label{eq:kink_perturbation}
\xi(r, \theta, z) = \xi(r) e^{i(m\theta + kz)},
\end{equation}
where $m$ and $k$ are the wavenumbers \rs{in the $\theta$ and $z$
  directions, respectively}{in the azimuthal and axial directions,
  $\theta$ and $z$, respectively and $r$ is the radial coordinate in
  cylindrical polars}. 
The helical kink instability occurs for perturbations where $m=1, k\ne0$ and is the only instability of this form which is a body instability, that is it moves the entire body of the flux tube. Perturbations where $m>1$ are termed flute or interchange instabilities.

When the magnetic field is sheared, as in a twisted magnetic flux
tube, an interchange instability (such as the flute
instability) is confined to a surface where the peaks and troughs
follow the shear of the field. That is, the instability is confined to
the surface where the perturbation wavevector $(0, m/r, k)$ is
perpendicular to the direction of the field, \rs{}{known as the
  ``resonance surface''}. In an axisymmetric twisted
flux tube the resonance surface is located at a radius $r$
given by
\begin{equation}
  \label{eq:resonant_surface}
\frac{m}{r} B_{\theta}(r) + kB_z(r) \approx 0.
\end{equation}

The stability of an infinite cylindrical flux tube to perturbations of
the form~\eqref{eq:kink_perturbation} is given by the classical
Suydam's criterion~\citep{suydamStabilityLinearPinch1958}
\begin{equation}
  \label{eq:suydams_criterion}
\frac{B_z^2 S^2}{4} + 2 r p' > 0,
\end{equation}
where $S = r q'/q$ is a measure of the shear, $q = 2\pi r B_z / L
B_{\theta}$ is the safety factor for a flux tube of length $L$ and a
prime denotes differentiation with respect to
$r$~\citep{mikhailovskiiInstabilitiesConfinedPlasma1998}. This applies
to both flute and kink instabilities although many
additional effects such as line-tying are not incorporated into the
corresponding linear analysis. The effect of line-tying on the kink
instability is investigated
in~\citet{hoodKinkInstabilitySolar1979}. Where~\eqref{eq:suydams_criterion}
is not satisfied, the flux tube may be unstable to perturbations of
the form~\eqref{eq:kink_perturbation}. When $m>1$, the perturbations
remain local to resonant surfaces given
by~\eqref{eq:resonant_surface}. 
When Suydam's criterion is satisfied and the flux tube is linearly stable, it may still be unstable to non-local perturbations, where the shear and pressure are small enough that interchange perturbations do not need to satisfy~\eqref{eq:resonant_surface}. Additionally, the inclusion of resistivity generally reduces the stabilising effect of the shear, permitting growth of a resistive interchange mode, albeit at a slower rate than that of the ideal instability~\citep{mikhailovskiiInstabilitiesConfinedPlasma1998}. It will be found that the ideal linear analysis of~\cite{mikhailovskiiInstabilitiesConfinedPlasma1998} is sufficient for understanding the flute instabilities investigated here since the associated flux tubes adequately fail the criterion~\eqref{eq:suydams_criterion}.

While Suydam's condition gives an indication of the stability of a flux tube to a given perturbation, the linear growth rate of the ideal flute instability $\gamma$ can be found via a stability analysis analogous to that of the Rayleigh-Taylor instability (see~\citep[][]{goldstonIntroductionPlasmaPhysics2020}) and is given by
\begin{equation}
  \label{eq:fluting_growth_rate}
\gamma^2 = \frac{2|\nabla p|}{\rho R_c},
\end{equation}
where $R_c$ is the radius of curvature of the magnetic field. This equation only applies when the pressure gradient and radius of curvature vector are in the same direction, that is the plasma is constrained by a concave magnetic field such that the pressure forces and magnetic tension forces are in competition. In a cylindrical, twisted flux tube, the field is always concave towards the central axis of the tube, so any inwardly directed pressure gradient is potentially unstable to fluting.

Throughout this paper, the twisted flux tube generated by the drivers has a pressure profile which is approximately axisymmetric, and independent of $z$ away from the boundaries at $z=\pm2$, and has a negative gradient, hence $|\nabla p|$ may be written as $-d p/ dr$. Similarly, away from the boundaries, the magnetic field has a negligible $r$ component and little dependence on $\theta$ and $z$, allowing the field to be approximated as $\vec{B} = (0, B_{\theta}(r), B_z(r))^{\text{T}}$, in cylindrical coordinates $(r, \theta, z)$. For a twisted field of this form, the radius of curvature is given by 
\begin{equation}
  \label{eq:radius_of_curvature}
  R_c = \frac{1}{|(\vec{b}\cdot\nabla) \vec{b}|} = \frac{r}{b_{\theta}^2},
\end{equation}
where $\vec{b} = \vec{B}/|\vec{B}|$ is the unit vector in the direction of the magnetic field and $b_{\theta}$ is the component of $\vec{b}$ in the azimuthal direction. These approximations allows the growth rate to be written as
\begin{equation}
  \label{eq:fluting_growth_rate2}
\gamma_{\text{ideal}}^2 = \frac{-2p'}{\rho R_c}.
\end{equation}
This approximation for the growth rate continues to hold while the flux tube remains relatively axisymmetric, that is while the kink instability remains in its linear phase.

\rs{In contrast to the precise form of the equilibrium state and
  perturbation studied in~\citet{quinnEffectAnisotropicViscosity2020},
  this is an experiment where a system is driven and instabilities
  occur organically as a result of noise providing a random
  perturbation. As a result of the driving, the flux tube is not in
  static equilibrium. Consequently, the stability criterion and linear
  growth rate presented previously are useful only as a guide and for
  approximate comparison.}{The stability criterion
  \eqref{eq:suydams_criterion} and the linear
  growth rate approximation \eqref{eq:fluting_growth_rate2} are useful
  only as a guide and for 
  approximate analysis of the numerical simulations presented in this
  work.  The precise form of the equilibrium state and the
  perturbations needed for the validity of
  \eqref{eq:suydams_criterion} and \eqref{eq:fluting_growth_rate2}
  were used in~\citet{quinnEffectAnisotropicViscosity2020}.
  In contrast, in the experiments reported in the following the system
  is driven and instabilities occur spontaneously due to random
  perturbation. As a result of the driving, the flux tube is also not
  in static equilibrium initially. }


\section{Mathematical formulation and numerical setup}
\label{sec-numerical-setup}

We consider the
magnetohydrodynamic equations for the density $\rho$, plasma
velocity $\vec{u}$, pressure $p$, magnetic field $\vec{B}$ and
internal energy $\varepsilon$, \rs{}{in their non-dimensionalised
  visco-resistive form}
\begin{subequations}
  \label{eq:MHD}
  \begin{gather}
%\begin{equation}
\label{eq:mhda}
\frac{D\rho}{Dt} = - \rho \vec{\nabla} \cdot \vec{u},\\
%\end{equation}
%\begin{equation}
\rho\frac{D\vec{u}}{Dt} = -\vec{\nabla} p + \vec{\jmath} \times \vec{B} + \vec{\nabla} \cdot \ten{\sigma},\\
%\end{equation}
%\begin{equation}
\frac{D\vec{B}}{Dt} = (\vec{B} \cdot \vec{\nabla})\vec{u} - (\vec{\nabla} \cdot \vec{u})\vec{B} + \eta \nabla^2 \vec{B},\\
%\end{equation}
%\begin{equation}
\rho\frac{D\varepsilon}{Dt} = -p \vec{\nabla} \cdot \vec{u} + {Q}_{\nu} + {Q}_{\eta},%\\
%\label{eq:energy}
%\end{equation}
    \end{gather}
\end{subequations}
where $\eta$ is the non-dimensionalised resistivity,
$\vec{\jmath} = \nabla \times \vec{B}$ is the current
density and \rs{$\ten{\sigma}$ is the viscous stress tensor (described
  below). The}{the} terms ${Q}_{\nu} = \ten{\sigma} : \vec{\nabla}\vec{u}$
  and ${Q}_{\eta} = \eta | \vec{\jmath} |^2$ are viscous heating and
  Ohmic heating, respectively. The system is closed by the inclusion
  of the equation of state for an ideal gas 
\begin{equation}
\varepsilon = \frac{p}{\rho(\gamma - 1)},
\end{equation}
with the specific heat ratio is given by $\gamma = 5/3$.

\rs{The isotropic (or Newtonian) viscosity stress tensor is given by
\begin{equation}
  \label{eq:isotropic_viscous_tensor}
  \ten{\sigma}_{\text{iso}} = \nu \ten{W},
\end{equation}
where $\nu$ is the viscous transport parameter, generally referred to as the viscosity,
\begin{equation}
  \label{eq:rate_of_strain}
  \ten{W} = \nabla\vec{u} + (\nabla\vec{u})^T - \tfrac{2}{3}(\nabla \cdot \vec{u})\ten{I},
\end{equation}
is the rate of strain tensor, and $\ten{I}$ is the  $3\times 3$
identity matrix. We use an identical model of anisotropic viscosity to
that used in
previous work~\citep{quinnEffectAnisotropicViscosity2020,quinnKelvinHelmholtzInstabilityCollapse2021},
 the switching model, first presented
 in~\citet{mactaggartBraginskiiMagnetohydrodynamicsArbitrary2017}. While
 this model is intended to revert to an isotropic model in the
 vicinity of magnetic null points, we effectively only make use of the
 anisotropic component, since there are no null points present in the
 simulations reported here at any point. To this
 end, we set the switching parameter (which controls the degree of
 viscous anisotropy) to $a_0 = 150$ in Lare3d's control, a value
 sufficiently large so that the anisotropic viscous stress tensor is given by
\begin{equation}
  \label{eq:switching_model}
\ten{\sigma}_{\text{swi}} = \nu \left[\frac{3}{2}(\ten{W}\vec{b}\cdot\vec{b}) \left( \vec{b} \otimes \vec{b} - \frac{1}{3}\ten{I} \right)\right],
\end{equation}
where $\vec{b}$ is the unit vector in the direction of the magnetic field. This tensor is identical to the strong field approximation of Braginskii~\citep{braginskiiTransportProcessesPlasma1965}. Full use of the switching model can be found in~\citet{mactaggartBraginskiiMagnetohydrodynamicsArbitrary2017,quinnEffectAnisotropicViscosity2020,quinnKelvinHelmholtzInstabilityCollapse2021}.
}{}

\rs{}{Two different models for the viscosity stress tensor $\ten{\sigma}$
will be compared and contrasted in this study. The first model is the
conventional} isotropic (or Newtonian) viscosity stress tensor \rs{}{used in
the vast majority of the existing literature, so that,} 
\begin{equation}
  \label{eq:isotropic_viscous_tensor}
\ten{\sigma = }\ten{\sigma}_{\text{iso}} = \nu \ten{W},
\end{equation}
where $\nu$ is the viscous transport parameter, generally referred to as the viscosity,
\begin{equation}
  \label{eq:rate_of_strain}
  \ten{W} = \nabla\vec{u} + (\nabla\vec{u})^T - \tfrac{2}{3}(\nabla \cdot \vec{u})\ten{I},
\end{equation}
is the rate of strain tensor, and $\ten{I}$ is the  $3\times 3$
identity tensor. The second model, which is the one of actual interest, is the anisotropic viscosity stress
tensor given by
\begin{equation}
  \label{eq:switching_model}
\rs{}{\ten{\sigma} = \ten{\sigma}_\text{aniso}} = \nu \left[\frac{3}{2}(\ten{W}\vec{b}\cdot\vec{b}) \left( \vec{b} \otimes \vec{b} - \frac{1}{3}\ten{I} \right)\right],
\end{equation}
where $\vec{b}$ is the unit vector in the direction of the magnetic
field.
\rs{}{
Expression \eqref{eq:switching_model} is identical to the strong field approximation of the
general anisotropic viscosity tensor derived in~\citet{braginskiiTransportProcessesPlasma1965}.
Expressions \eqref{eq:isotropic_viscous_tensor} and \eqref{eq:switching_model} arise as asymptotic limits of the
more general switching model used in our earlier works 
\citep{mactaggartBraginskiiMagnetohydrodynamicsArbitrary2017,quinnEffectAnisotropicViscosity2020,quinnKelvinHelmholtzInstabilityCollapse2021}
which, includes both isotropic and anisotropic contributions and can
switch gradually between them depending on the strength of the
magnetic field at a given spacio-temporal location. For example, in
the vicinity of a null point where the magnetic field becomes weak the
isotropic viscosity contribution becomes dominant in the switching
model. Switching between the two limit cases is not relevant in the
present study where the variations in the magnetic field are not
significantly large.}

The non-dimensionalisation of equations \eqref{eq:MHD} is identical to that used in
our earlier works~\citep{quinnEffectAnisotropicViscosity2020,quinnKelvinHelmholtzInstabilityCollapse2021}. A typical magnetic
field strength $B_0$, density $\rho_0$ and length scale $L_0$ are
chosen and the other variables non-dimensionalised
appropriately. Velocity and time are non-dimensionalised using the
Alfv\'en speed $u_A = B_0 / \sqrt{\rho_0 \mu_0}$ and Alfv\'en crossing 
time $t_A = L_0/u_A$, respectively. Temperature is non-dimensionalised
via $T_0 = u_A^2 \bar{m} / k_B$, where $k_B$ is the Boltzmann constant
and $\bar{m}$ is the average mass of ions, here taken to be $\bar{m} =
1.2m_p$ (a mass typical for the solar corona) where $m_p$ is the
proton mass. Dimensional quantities can be recovered by multiplying
the non-dimensional variables by their respective reference value
(e.g. $\vec{B}_{\dim} = B_0 \vec{B}$). The reference values used here
are $B_0 = 5 \times 10^{-3}$ T, $L_0 = 1$ Mm and $\rho_0 = 1.67 \times
10^{-12} \ \text{kg m}^{-3}$, giving reference values for the Alfv\'en
speed $u_A = 3.45\ \text{Mm s}^{-1}$, Alfv\'en time $t_A =
0.29\ \text{s}$ and temperature $T_0 = 1.73 \times 10^{9}\ K$.      

\rs{}{The following initial and boundary conditions are used to form a
magnetic flux tube and excite instabilities by dynamic twisting.} 
The magnetic field is prescribed as initially straight and uniform,
\begin{equation}
\vec{B} = (0, 0, 1)^{\text{T}},
\end{equation}
in a cube of size $[-2,2]^3$, with further test simulations run using an elongated domain of size $4\times4\times20$. Initially, the velocity is set
everywhere to $\vec{u} = \vec{0}$, the density to $\rho = 1$,
and the
internal energy to $\varepsilon = 8.67\times 10^{-4}$. This corresponds
to a typical coronal temperature of $10^6$ K and a plasma beta of $1.11 \times 10^{-4}$. At the
boundaries, the magnetic field, velocity, density, and energy are fixed to their
initial values \rs{}{and their derivatives normal to the
boundaries are set to zero}  except
where twisting velocity ``driver'', described
below, is prescribed.
\rs{The spatial
derivatives of these variables are also set to zero at the
boundaries.}{} \rs{The resolution is $512$ grid points per dimension,
comparable to the highest resolution kink instability studies
of~\cite{hoodCoronalHeatingMagnetic2009} or medium resolution studies
of~\cite{barefordShockHeatingNumerical2015}.}{}

\begin{figure}
  \centering
  \begin{subfigure}{.49\textwidth}
  \centering
  \includegraphics[width=1.0\linewidth]{u_r.pdf}
  \caption{Radial dependence of driver}
  \label{fig:kink_radial_driver}
  \end{subfigure}
  \begin{subfigure}{.49\textwidth}
  \centering
  \includegraphics[width=1.0\linewidth]{u_t.pdf}
  \caption{Acceleration of driver}
  \label{fig:kink_driver_accel}
  \end{subfigure}
  \mycaption{Radial velocity profile $u_r(r)$ and acceleration profile
    $u_t(t)$ of the driver \eqref{eq:null_twisting_profile} for
    parameters $u_0 = 0.15$, $r_d = 5$ and $t_r = 2$.}{} 
  \label{fig:kink_driver}
\end{figure}

The flux rope is formed by prescribing a slowly accelerating, rotating flow at the upper $z$-boundary as
\begin{equation}
  \label{eq:null_twisting_profile}
  \vec{u} = u_0 u_r(r) u_t(t) (-y, x, 0)^T,
\end{equation}
where $u_r(r)$ describes the radial profile of the twisting motion in terms of the radius $r^2 = x^2 + y^2$,
\begin{equation}
  \label{eq:radial_twisting_function}
  u_r(r) = u_{r0}(1 + \tanh(1 - r_d r^2)),
\end{equation}
where $r_d$ controls the radial extent of the driver, $u_{r0}$ is a normalising factor, and $u_t(t)$ describes the imposed acceleration of the twisting motion,
\begin{equation}
  \label{eq:ramping_up_function}
  u_t(t) = \tanh^2(t/t_r),
\end{equation}
where the parameter $t_r$ controls the time taken to reach the final driver velocity $u_0$. The functions $u_r(r)$ and $u_t(t)$ are plotted in figure~\ref{fig:kink_driver}. At the lower boundary, the flow is in the opposite direction. This form of driver allows the system to be accelerated slowly enough that the production of disruptive shocks and fast waves is minimal. It is unavoidable that some waves are produced during the boundary acceleration, however these usefully provide a source of noise which eventually forms a perturbation.
\rs{}{
The driver} velocity is set to $u_0 = 0.15$, the normalising factor is $u_{r0} = 2.08$, and setting $r_d = 5$ corresponds to a driver constrained to $r<1$ and with a peak velocity at $r\approx 0.38$. The ramping time is set to $t_r = 2$, resulting in an acceleration from $0$ to $u_0$ over approximately $5$ Alfv\'en times. These driver parameters correspond to a peak rotational period of $T_R = 15.92$, the length of time taken for one full turn to be injected by a single driver. Both drivers result in twist being added at a rate of $2\pi$ every $7.96$ Alfv\'en times. The twist profile across the entire flux tube develops in such a way that by $t\approx 20$, it is qualitatively similar to those studied in~\citet{quinnEffectAnisotropicViscosity2020,hoodCoronalHeatingMagnetic2009,barefordShockHeatingNumerical2015} however the length of the flux tubes differs significantly. This configuration produces a $z$-directed tube of increasingly twisted magnetic field that eventually becomes unstable to both the flute instability and the helical kink instability.

\rs{}{The problem formulated above is solved numerically using the
  staggered-grid, Lagrangian–Eulerian remap code for 3D MHD
  simulations Lare3D of~\cite{arberStaggeredGridLagrangian2001} where
  a  new module for anisotropic viscosity has been
  included as detailed in~\citet{quinnKelvinHelmholtzInstabilityCollapse2021}.} The resolution used in the
current work is  $512$ grid points per dimension, comparable to the
highest resolution kink instability studies
of~\cite{hoodCoronalHeatingMagnetic2009} or medium resolution studies
of~\cite{barefordShockHeatingNumerical2015}. 

\section{Results}
\label{sec-results}

\rs{Two pairs of simulations were performed,}{We focus the attention
on two selected pairs of  
simulations, one pair where the background resistivity is set to
$\eta=10^{-3}$ and another where $\eta=10^{-4}$. As
in~\citet{quinnEffectAnisotropicViscosity2020}, only background
resistivity is used. Each pair consists of one simulation
using isotropic viscosity~\eqref{eq:isotropic_viscous_tensor} and
another one using the anisotropic model~\eqref{eq:switching_model}. The
value of viscosity is set to $\nu = 10^{-4}$ in all cases.} 
%
The overall development of both the flute and the kink
instabilities is broadly similar for the two values of resistivity and
is described  initially. Similar simulations were performed with a longer flux tube of length $20$ instead of the tubes shown here with length $4$, and the results were found to be qualitatively similar. Focus is 
then placed on the detailed description of instabilities in the
$\eta=10^{-4}$ cases, with the aim of comparing the effects of the two
viscosity models. \rs{These cases illustrate the development of the instabilities while showcasing the effect of the
viscosity models.}{} Then further features of the $\eta=10^{-3}$
cases are summarised. \rs{without a full exploration of the (qualitatively
similar) results.}{}

\subsection{\rs{Overview of instability development}{Mechanism and
    general development of instability}}

\begin{figure*}
  \centering
    \begin{subfigure}{0.32\textwidth}
      \includegraphics[width=\linewidth]{current_profiles.pdf}
      \caption{Current}
      \label{fig:current_profiles}
    \end{subfigure}
    \hfill
    \begin{subfigure}{0.32\textwidth}
      \includegraphics[width=\linewidth]{ohmic_heating_profiles.pdf}
      \caption{Ohmic heating}
      \label{fig:kink_straight_ohmic_heating_profile}
    \end{subfigure}
    \hfill
    \begin{subfigure}{0.32\textwidth}
      \includegraphics[width=\linewidth]{pressure_profiles.pdf}
      \caption{Pressure}
      \label{fig:pressure_profiles}
    \end{subfigure}
  \mycaption{Gradients in the current density generate pressure
    gradients through Ohmic heating.}{\rs{}{Axial current density (a), Ohmic heating
      (b) and pressure (c) as functions of the radial distance from
      the tube axis.}{} All plots are from anisotropic
    cases when $t=20$ through the plane $z=0$. Note the sign of the
    axial current density $j_z$ has been flipped for comparison and
    the Ohmic heating is given by $Q_{\eta} = \eta j^2$. The pressure
    profile of an additional test-case where $\eta=0$ is also
    shown. Line types are indicated in the legend.}%
  \label{fig:pressure_and_heating}
\end{figure*}

\rs{In all cases, the initial reaction to the twisting at the upper
and lower boundaries is two}{Initially and in all cases computed, the
twisting at the upper and lower boundaries gives rise to a pair of}
torsional Alfv\'en waves which proceed to travel along the tube
from the upper and lower boundaries to their respective
opposite boundaries. The interaction between these waves produces an 
oscillating pattern in the kinetic energy with a period of
approximately $4$ Alfv\'en times, equal to the time taken for an
Alfv\'en wave to traverse the entire length of the domain
as visible early in figure~\ref{fig:kink_ke-4}. 

As the field continues to be twisted, currents form, due to the local
shear in the magnetic field, and heat the plasma through Ohmic
dissipation. Due to the radial form of the driver, the magnitude of the
current density is greatest at the axis of the tube, then decreases
radially outwards as seen in figure~\ref{fig:current_profiles}. The orientation
of the twisting produces a current flowing in the negative
$z$-direction for $r\lessapprox0.5$. At $r \approx 0.5$ (corresponding
to the radius of peak driving velocity) the current switches orientation
and is in the positive $z$-direction in a shell where $0.5\lessapprox
r \lessapprox 0.8$. This form of a twisted field with an inner core of
current in one direction surrounded by a shell of
oppositely-directed current is similar to the current configuration arising due to the field prescribed in~\citet{quinnEffectAnisotropicViscosity2020}. 

This current profile is reflected in the radial Ohmic heating profile
(figure~\ref{fig:kink_straight_ohmic_heating_profile}) and,
consequently, in the radial pressure profile
(figure~\ref{fig:pressure_profiles}). The highly pressurised core
extends to $r\approx 0.2$--$0.4$ (depending on the value of $\eta$)
before increasing slightly around $r\approx 0.7$. The secondary bump
in pressure is due to the outer shell of current. The pressure
gradient near the axis provides the outwardly directed pressure force
which competes against the binding action of the magnetic tension
and this provides the mechanism of
flute instability excitation.  The magnitude
of the pressure gradient depends strongly on the value of
resistivity $\eta$, with lower values producing shallower 
gradients which (as shall be seen) are more stable to the
flute instability. Indeed, when $\eta=0$, the radial
pressure profile is nearly flat and the tube stable to the
flute instability.

In all cases unstable to the flute instability, it
occurs  between $t=20$ and $t=30$. \rs{During this
  time, the}{The} continued driving at the boundaries eventually
injects enough twist that the tube also becomes unstable to the kink
instability. This initially develops linearly alongside \rs{}{or shortly
after} the flute instability and then erupts during the
its  nonlinear phase, dominating the dynamics and
disrupting the   flute instability. The \rs{}{onset and
the competition} of the two instabilities is strongly affected by
the value of $\eta$ and the viscosity model used. 

\subsection{Instabilities at resistivity $\eta=10^{-4}$}

\begin{figure*}
  \centering
    \begin{subfigure}{0.32\textwidth}
      \includegraphics[width=\linewidth]{swi-4_pressure_13.pdf}
      \caption{$t=26$}
      \label{fig:swi-4_pressure_13}
    \end{subfigure}
    \hfill
    \begin{subfigure}{0.32\textwidth}
      \includegraphics[width=\linewidth]{swi-4_pressure_14.pdf}
      \caption{$t=28$}
      \label{fig:swi-4_pressure_14}
    \end{subfigure}
    \hfill
    \begin{subfigure}{0.32\textwidth}
      \includegraphics[width=\linewidth]{swi-4_pressure_15.pdf}
      \caption{$t=30$}
      \label{fig:swi-4_pressure_15}
    \end{subfigure}
\mycaption{Pressure profiles during the development of the
  flute and kink instabilities.}{Shown are
  density plots of pressure at $z=0$
  with $\eta = 10^{-4}$ 
  and for the anisotropic viscosity model}. Note the difference in colour scale
  in figure~\ref{fig:swi-4_pressure_15}. The development in the
  isotropic case is similar. 
\label{fig:kink_pressure_slices-4}%
\end{figure*}

\rs{}{We now describe the evolution and competition of flute and
  kink instabilities in case of resistivity $\eta=10^{-4}$.}
Figure~\ref{fig:kink_pressure_slices-4} shows the pressure profile
\rs{the viscosity model is switching for times}{of the \rs{switching}{anisotropic} viscosity case 
\eqref{eq:switching_model} at
time moments} $t=26$, $28$ and $30$ and at $z=0$. At $t=26$
the tube becomes unstable to  flute
instability with azimuthal wavenumber $m=4$, when the
plasma begins to  bulge out diagonally from the
high-pressure core as seen in figure~\ref{fig:swi-4_pressure_13}. As  
the bulges move radially outwards into lower pressure regions they
expand and accelerate, resulting in the entire pressure structure
appearing taking the shape of a four-leaf clover (figure~\ref{fig:swi-4_pressure_14}). By
$t=30$ the kink instability has disrupted the flute
instability and is developing nonlinearly
as evident in figure~\ref{fig:swi-4_pressure_15}. As is typical of nonlinear kink
development, the tube continues to release its stored potential energy
in the form of kinetic energy and heat and the contained plasma becomes highly
mixed. In the  isotropic viscosity case
which will not be illustrated by a separate figure, the
flute instability is present but damped, and \rs{}{it is
  quickly outcompeted by} the kink instability which quickly dominates the dynamics. 

\begin{figure}
  \centering
    \begin{subfigure}{0.49\textwidth}
      \includegraphics[width=\linewidth]{field_line_plots/cropped/v-4r-4-isotropic_0014_cropped.png}
      \caption{Isotropic}
      \label{fig:field_line_plots_iso}
    \end{subfigure}
    \hfill
    \begin{subfigure}{0.49\textwidth}
      \includegraphics[width=\linewidth]{field_line_plots/cropped/v-4r-4-switching_0014_cropped.png}
      \caption{Anisotropic}
      \label{fig:field_line_plots_swi}
    \end{subfigure}
\mycaption{Simultaneous development of flute and kink
  instabilities in the isotropic and anisotropic cases
  illustrated by field lines and pressure contours.}{Field
  lines and contours of pressure (where $p=0.3$) are plotted at
  $t=28$. Also shown is the velocity driver \rs{as a slice}{$u_r(\sqrt{x^2+y^2})$ at $z=2$}. The flute instability grows in both cases, though faster in the anisotropic case. The initial stages of the kink instability can also be observed in the field lines of the isotropic case in subfigure~\ref{fig:field_line_plots_iso}.}
\label{fig:kink_field_line_plots}%
\end{figure}

Figure~\ref{fig:kink_field_line_plots} shows the effect the viscosity
models have on the initial stages of the flute and kink
instabilities in 3D. While the flute instability is
observed in both cases, it is damped in the isotropic case and grows
faster in the anisotropic case. In the latter case,
the extended development of the flute instability
appears to disrupt the inner core of field lines and, \rs{(as shall be
  seen)}{as will be discussed further below,} slows the growth of the
kink instability. In the isotropic case, the flute
instability has been damped to the extent that the kink instability
grows uninhibited and quickly disrupts the fluting. 

\begin{figure}
  \centering
    \begin{subfigure}{0.49\textwidth}
      \includegraphics[width=\linewidth]{kinetic_energy-4.pdf}
      \caption{Kinetic Energy}
      \label{fig:kink_ke-4}
    \end{subfigure}
    \hfill
    \begin{subfigure}{0.49\textwidth}
      \includegraphics[width=\linewidth]{kinetic_energy_log-4.pdf}
      \caption{Growth rate estimation}
      \label{fig:kink_ke_log-4}
    \end{subfigure}
  \mycaption{Kinetic energy as a function of time showing the
    development and measured growth rates $\gamma$ and $\lambda$
    of the flute
    and kink instabilities, respectively.}{\rs{Both plots are from results
      where}{Resistivity is $\eta=10^{-4}$ and the second plot is a
      enlarged version of the first.}}
\label{fig:kink_str8_ke-4}%
\end{figure}

Despite the flute instability appearing in the isotropic
case (figure~\ref{fig:field_line_plots_iso}), only in the
anisotropic case can the onset of both the
flute and kink instabilities be seen in the kinetic
energy profile shown in figure~\ref{fig:kink_ke_log-4}. Here, the nonlinear
growth rates of the two instabilities are found to be $\gamma = 0.69$
for the flute and $\lambda = 2.55$ for the kink. The
onset times are approximately $t=27$ for the flute
instability and $t=29.5$ for
the kink. In the isotropic case, the growth rate of the kink, $\lambda
= 2.97$, is larger than in the anisotropic case,
although the onset times appear similar, and the kinetic energy
profile shows no evidence of \rs{the growth of the flute
instability.}{flute instability growth.} 

The faster growth of the kink compared to that measured in~
\citet{quinnEffectAnisotropicViscosity2020} is attributed to
the relative aspect ratios of the flux tubes. The tube prescribed in
~\citep{quinnEffectAnisotropicViscosity2020} has an aspect
ratio of approximately $20$ compared to the tube studied here which
has an aspect ratio of approximately $4$. While the total twist is
similar in both tubes (after the drivers have injected twist up to
$t\approx20$) the small aspect ratio results in more turns per unit
length, leading to a faster growing instability. 

\begin{figure}
  \centering
    \begin{subfigure}{0.49\textwidth}
      \includegraphics[width=\linewidth]{suydam_condition_4.pdf}
      \caption{Suydam condition}
      \label{fig:suydam_condition_4}
    \end{subfigure}
    \hfill
    \begin{subfigure}{0.49\textwidth}
      \includegraphics[width=\linewidth]{growth_rate_4.pdf}
      \caption{Linear growth rate}
      \label{fig:growth_rate_4}
    \end{subfigure}
\mycaption{Stability and linear growth rate of the flute
  instability.}{In panel \ref{fig:suydam_condition_4}, Suydam's stability
  criterion \eqref{eq:suydams_criterion} and is contributing
  terms   are plotted and in
  panel \ref{fig:growth_rate_4} the predicted linear growth rate~\eqref{eq:fluting_growth_rate2} is plotted. Both plots are produced at $t=20$ for $\eta=10^{-4}$ and using the anisotropic model. The location of the peak linear growth rate is also shown.}
\label{fig:stability_and_growth}%
\end{figure}

Prior to the onset of either instability, the flux tube is found to be
linearly unstable to perturbations of the
form~\eqref{eq:kink_perturbation} at $t=20$ via Suydam's
criterion~\eqref{eq:suydams_criterion}
(figure~\ref{fig:suydam_condition_4}). The criterion represents a
balance between destabilising pressure gradients and stabilising
magnetic shear and in this case, the shear is so small and the
pressure gradient so large that the tube is unstable over a wide range
of radii, for $ 0.02 \lessapprox r \lessapprox 0.29$. The measure of
linear fluting growth rate $\gamma$ is plotted as a function of $r$ at
the same time (figure~\ref{fig:growth_rate_4}). The location of the peak growth matches nearly exactly the location of the resonant surface where the observed perturbation grows (figure~\ref{fig:swi-4_pressure_13}) and an estimate of the linear growth rate can be found by averaging $\gamma$ over $r$, giving a growth rate of $0.86$.

\begin{figure}
  \centering
    \begin{subfigure}{0.49\textwidth}
      \includegraphics[width=\linewidth]{perturbations_4.pdf}
      \caption{Perturbations}
      \label{fig:pressure_pert_4}
    \end{subfigure}
    \hfill
    \begin{subfigure}{0.49\textwidth}
      \includegraphics[width=\linewidth]{resonant_surface_4.pdf}
      \caption{Resonance function}
      \label{fig:resonant_surface_4}
    \end{subfigure}
\mycaption{Perturbations corresponding to the flute and
  kink instabilities and the spatial radial distribution of the
  associated resonance function.}{Pressure and velocity perturbations
  in $z$ (corresponding to the flute and kink
  instabilities, respectively) and of the resonance
  surface $m B_{\theta}(r)/r + kB_z(r)$ as a function
  of $r$ using the observed fluting perturbation wavenumbers. All
  plots are  snapshots at $t=26$ where $\eta=10^{-4}$ and
  the viscosity model is anisotropic.} 
\label{fig:k_and_resonance}%
\end{figure}

The observed
  perturbations corresponding to the flute and kink
  instabilities at $t=26$ \rs{}{are shown in
  Figure~\ref{fig:pressure_pert_4}}. The fluting perturbation is
  most easily observed in the pressure and is plotted as a function of $z$
  following a line through the point $(r, \theta) = (0.101, 0)$. The
  kink instability is best revealed in the $x$-velocity (a proxy for the
  radial velocity) through the axis. Comparing the magnitudes of the
  perturbations at this time suggests the flute
  instability is close to transitioning to its nonlinear phase while
  the kink instability is still very much in its linear
  phase.

The value of $k$ for each perturbation is calculated as $k = 2\pi/\tilde{\lambda}$ where $\tilde{\lambda}$ is the wavelength of the perturbation, measured as the difference between the two peaks closest to $z=0$ (thus minimising the influence of line-tying on the measurement). This gives a value of $k_{\text{flute}}=23.61$ and $k_{\text{kink}}=4.57$ for both viscosity models. Hence, the observed most unstable fluting perturbation is that of the form~\eqref{eq:kink_perturbation} where $m=4$ and $k=23.61$ and the observed kink instability is that where $m=1$ and $k=4.57$. Using these values, it is observed that the fluting perturbation exactly resonates with the field, that is $m B_{\theta}(r)/r + kB_z(r) = 0$, at $r=0.125$ (figure~\ref{fig:resonant_surface_4}). This is precisely the predicted radius of peak linear growth (figure~\ref{fig:growth_rate_4}). At this time the perturbation is close to resonance, that is $m B_{\theta}(r)/r + kB_z(r) \approx 0$, over a range of radii from $r=0.125$ to $0.2$.

Comparing the effect of the viscous models on the perturbations, in the isotropic case, the fluting perturbation is damped, while in the anisotropic case the kink perturbation is diminished, explaining why the flute instability appears more readily in the anisotropic case (figure~\ref{fig:kink_ke-4}).

\subsection{Instabilities at resistivity $\eta=10^{-3}$}

\begin{figure*}
  \centering
    \begin{subfigure}{0.32\textwidth}
      \includegraphics[width=\linewidth]{swi-3_pressure_12.pdf}
      \caption{$t=24$}
      \label{fig:swi-3_pressure_12}
    \end{subfigure}
    \hfill
    \begin{subfigure}{0.32\textwidth}
      \includegraphics[width=\linewidth]{swi-3_pressure_14.pdf}
      \caption{$t=28$}
      \label{fig:swi-3_pressure_14}
    \end{subfigure}
    \hfill
    \begin{subfigure}{0.32\textwidth}
      \includegraphics[width=\linewidth]{swi-3_pressure_15.pdf}
      \caption{$t=30$}
      \label{fig:swi-3_pressure_15}
    \end{subfigure}
    \hfill
    \begin{subfigure}{0.32\textwidth}
      \includegraphics[width=\linewidth]{swi-3_pressure_16.pdf}
      \caption{$t=32$}
      \label{fig:swi-3_pressure_16}
    \end{subfigure}
    \hfill
    \begin{subfigure}{0.32\textwidth}
      \includegraphics[width=\linewidth]{swi-3_pressure_17.pdf}
      \caption{$t=34$}
      \label{fig:swi-3_pressure_17}
    \end{subfigure}
    \hfill
    \begin{subfigure}{0.32\textwidth}
      \includegraphics[width=\linewidth]{swi-3_pressure_18.pdf}
      \caption{$t=36$}
      \label{fig:swi-3_pressure_18}
    \end{subfigure}
\mycaption{Pressure profiles at $z=0$ during the development of the flute and kink instabilities in the higher resistivity anisotropic case.}{The viscosity model is anisotropic and $\eta = 10^{-3}$. In contrast to the case of $\eta=10^{-4}$, the nonlinear development of the flute instability has time to mix the interior of the flux tube before the onset of the kink instability, the growth of which is affected by the mixed plasma.}
\label{fig:kink_pressure_slices-3}%
\end{figure*}

\rs{Figures \ref{fig:kink_pressure_slices-3} show the}{Figure
  \ref{fig:kink_pressure_slices-3} shows a} prolonged 
development of the flute instability and a slow
nonlinear development of the kink \rs{}{instability at the higher
  resistivity value $\eta=10^{-3}$ in the case of anisotropic viscosity}. Due to the enhanced Ohmic heating 
at $\eta=10^{-3}$, the pressure gradient is substantially stronger
than at $\eta=10^{-4}$ and the flute instability is
excited much earlier. Compared to the $\eta=10^{-4}$ cases, the
instability occurs further from the axis, at $r\approx0.16$, and the
larger pressure gradient drives the bulges in profile further
from the axis during the nonlinear phase
(figure~\ref{fig:swi-3_pressure_12}). These bulges continue to extend
outwards and mix the plasma as they develop. The kink instability can
be observed moving the axis of the tube diagonally upwards and to the
right in figure~\ref{fig:swi-3_pressure_15}. At this time in the
$\eta=10^{-4}$ cases, the nonlinear development of the kink was
at a later stage of its development (figure~\ref{fig:swi-4_pressure_15}). The development of
the kink then proceeds slowly as it moves the axis of the tube through
the mixed region to eventually begin the reconnection process with the
outer region of field that is typical of the instability in this kind
of flux tube (as was observed in
our earlier work~\citep{quinnEffectAnisotropicViscosity2020}). 

\begin{figure}
  \centering
    \begin{subfigure}{0.49\textwidth}
      \includegraphics[width=\linewidth]{kinetic_energy-3.pdf}
      \caption{Kinetic Energy}
      \label{fig:kink_ke-3}
    \end{subfigure}
    \hfill
    \begin{subfigure}{0.49\textwidth}
      \includegraphics[width=\linewidth]{kinetic_energy_log-3.pdf}
      \caption{Growth rate estimation}
      \label{fig:kink_ke_log-3}
    \end{subfigure}
\mycaption{Kinetic energy as a function of time in the cases where $\eta=10^{-3}$.}{The results from both viscosity models are shown. The flute instability appears earlier than where $\eta=10^{-4}$ and the growth rate of the kink instability is decreased.}
\label{fig:kink_str8_ke-3}%
\end{figure}

It is evident from the kinetic energy profile that the flute instability develops much earlier than in the $\eta=10^{-4}$ cases and grows at an increased rate of $\gamma = 1.06$ (figure~\ref{fig:kink_ke_log-3}). The kink instability grows at a rate of $\lambda \approx 0.15$, much slower than that observed in the $\eta=10^{-4}$ cases, and much lower than the flute instability. One key observation is that, despite the early and disruptive growth of the flute instability, the kink instability still generates the bulk of the kinetic energy (figure~\ref{fig:kink_ke-3}).

Due to the influence of the drivers on the kinetic energy, the fluting growth rate is difficult to estimate from the kinetic energy profile as accurately as in the $\eta=10^{-4}$ cases. Since the kink instability occurs after the development of the fluting, its growth rate is similarly difficult to gauge. Nevertheless, it is clear that the flute instability grows at a rate of the same order as that in the $\eta=10^{-4}$ cases. It is also apparent that the kink instability grows much slower in the $\eta=10^{-3}$ cases.

\begin{table}
\caption{Quantitative differences in the observed perturbations
  between results \rs{for both values of}{between different
    resistivity values} $\eta$. \rs{}{In all measurements the
  anisotropic viscosity model is used except for $k_\text{kink}$ where the
  isotropic viscosity model us used. Measurements times are listed in
  the main text.}}
\centering
\begin{tabular}{ccc}
&
$\eta=10^{-4}$ &
$\eta=10^{-3}$ \\
\midrule
Theoretical average linear growth rate of flute $\gamma$ & 0.86 & 1.79  \\
Observed nonlinear growth rate of flute $\gamma$ & 0.69 & 1.06  \\
Observed growth rate of kink $\lambda$ & 2.55 & 0.15\\
\midrule
Theoretical radius $r_s$ of peak flute growth rate & 0.125 & 0.163 \\
Observed radius $r_s$ of peak flute growth rate & 0.125 & 0.163 \\
\midrule
Observed axial wave number $k_{\text{flute}}$ & 23.61 & 16.05 \\
Observed axial wave number $k_{\text{kink}}$ & 4.57 & 4.53 \\
\end{tabular}
\label{tab:kink_fluting_params}
\end{table}

Table~\ref{tab:kink_fluting_params} summarises the quantitative
differences between the results for the two values of \rs{}{the
  resistivity} $\eta$. All values are calculated from simulations using the anisotropic model with the exception of $k_{\text{kink}}$ which is measured from isotropic results due to noise in the anisotropic case (the value of $k_{\text{kink}}$ appears similar, however). The results of the isotropic cases are qualitatively similar. The radius of peak $\gamma$ is calculated at time $t=20$. The fluting wavenumber $k$ and observed $r_s$ are measured at times just prior to the nonlinear development of the flute instability, that is at $t=22$ when $\eta=10^{-3}$ and $t=26$ when $\eta = 10^{-4}$. The kink wavenumber is measured at $t=26$ in both cases. These times allow fair comparison between measurements.

The longitudinal wavenumber $k_{\text{kink}}$ of the observed kink perturbation remains similar in both cases since the instability is essentially governed by the twist injected by the driver which remains the same in both cases. In contrast, the longitudinal wavenumber $k_{\text{flute}}$ of the observed fluting perturbation is lower in the $\eta=10^{-3}$ cases. This is due to the different resonant surface within which the perturbation grows, the location being dictated by the peak of the linear growth rate. Note that the location of this peak again matches well the location of the observed resonant surface, as in the $\eta=10^{-4}$ cases. Similar to the $\eta=10^{-4}$ cases, the peak growth rate predicted by the linear analysis is the same order of magnitude as the observed growth rate.

\section{Discussion}
\label{sec-discussion}

It is
likely that the $m=4$ azimuthal perturbation is excited due to
influences from the boundaries in the Cartesian box, for example
through the interaction of reflected fast waves generated in part by
the driver. Performing a similar experiment in a cylindrical numerical
domain, or prescribing a variety of perturbations in the Cartesian
domain may reveal other, faster growing modes. The modes may also be
influenced by nonlinear coupling between the $m>1$ and $m=1$ modes, as
is found in the study of kink and flute
oscillations~\citep{terradasEffectMagneticTwist2018,rudermanNonlinearGenerationFluting2017}. 

As the current distribution, which develops as the flux tube is twisted, is similar to that found in the initial flux tube configuration of~\cite{quinnEffectAnisotropicViscosity2020},the question arises why the fluting instability is not observed in the latter. Although the current distribution (and thus heating and pressure distributions) in the tubes of~\cite{quinnEffectAnisotropicViscosity2020} may support the flute instability, the tube is initially perturbed with a motion close to an unstable eigenmode of the kink instability, resulting in the instability growing from $t=0$. In contrast, in the tubes studied here, such a perturbation must build from numerical noise, allowing a secondary, fluting perturbation to also build and become significant enough to observe.

Our set of numerical experiments has
shown that the mixing as a result of the nonlinear flute
instability appears to slow the growth of the kink instability. \rs{In
  the linear regime it seems unlikely that}{It seems unlikely that in
  the linear regime} the  perturbations of  the
flute and kink are able to directly couple, given that
the kink instability generally presents at the axis of a flux tube and
the flute at some resonant surface away from the
axis. Further investigation of the nonlinear interaction between the
two instabilities is required. 

Since the main driver of the flute instability is the
pressure gradient generated through Ohmic heating, it is prudent to
ask if the same pressure gradient could be generated using physical
coronal values of the resistivity, which are estimated to be
approximately
$\eta=10^{-8}$~\citep{craigAnisotropicViscousDissipation2009a},
and are thus much smaller than those studied
here. Additionally, the simulations presented here do not incorporate
radiation or thermal conduction, two processes which would remove
energy (and hence reduce pressure) from high-pressure regions in a coronal
loop and thus could prevent meeting the required
conditions for the growth of a 
flute instability. Indeed, at $\eta=10^{-4}$ the
flute instability was more quickly outcompeted by the kink
instability and appeared to have little impact on the resultant
dynamics, which mirror those of other kink instability studies, \rs{such
as~}{e.g.~}\citep{hoodCoronalHeatingMagnetic2009}. This suggests that even
lower values of resistivity would result in flux tubes without any
significant flute instability, at least for this form of
driver and mechanism of pressure generation. Regardless, coronal loops
with strong radial pressure gradients have been
observed~\citep{foukalTemperatureStructurePressure1975}, and such loops
may be unstable to the flute instability. Modelling of a
prescribed flute-unstable flux tube, as opposed to the dynamically
stressed loop investigated here, would provide a useful comparison to
observations, however it may be difficult to prescribe a tube which is
not also susceptible to kinking. Linear stability analyses of this
kind of flux tube (a dynamically created zero total axial current
tube) focus on the kink
instability~\citep{browningSolarCoronalHeating2003b} so do not provide
much insight into the potential for fluting without a kink. 

While our results show that a flux tube can be unstable to
the flute instability and yet the faster growing kink
instability can quickly dominate when the pressure gradient is small
enough, the opposite case is also observed\rs{,
  where}{. A faster} growing flute instability appears to slow the growth of
the kink instability although, importantly, it does not fully disrupt
the development of the kink. Understanding the balance between the
nonlinear growth rates of the two instabilities is important \rs{in
understanding}{for prediction of} whether the flute instability may be found
at all in the real solar corona, or whether its realistic growth rate is
too slow compared to that of the kink instability. 

\section{Conclusion}
\label{sec-conclusions}

This paper details the nonlinear development of two
ideal instabilities, the kink and the flute instabilities,
both of which develop naturally in the course of twisting an initially
straight magnetic flux tube. This provides a different approach to
that employed in the simulations performed in \rs{chapter}{our earlier
  study}~\citep{quinnEffectAnisotropicViscosity2020} in that the
instabilities are not excited by any prescribed perturbations but,
instead, the field is dynamically driven into an unstable state and
the perturbations provided by noise in the system. Not only is the
kink instability excited due to the twist in the field, \rs{}{but also
and near simultaneously} a pressure-driven flute
instability can also be excited in unstable pressure gradients
generated by Ohmic heating. Simulations 
were performed with two values of resistivity,
$\eta=10^{-3}$ and $10^{-4}$, and for two forms of viscosity, isotropic and
\rs{switching}{anisotropic}\rs{, providing}{. The results prove}an
initial and important first step \rs{into the simulation of}{towards
understanding} nonlinear flute instabilities in the
solar corona.  

It has been shown that the flute instability can be quickly dominated by the kink instability if the kink grows substantially faster than the flute. However, if the flute has time to develop nonlinearly, it mixes the plasma within the flux tube, generating small scale current sheets and releasing some magnetic energy. The overall effect of this mixing is to slow the growth of the kink instability. The slowed growth of the kink does not appear to significantly impact the kinetic energy released during its evolution, only the time over which it is released. 

The form of viscosity has been found to significantly affect the
growth of the flute instability. Importantly, isotropic
viscosity is found to damp the growth of the flute
instability to the degree that it is unable to grow appreciably before
the onset of the faster growing nonlinear kink instability. Overall,
the anisotropic model permits greater release of
kinetic energy. Similar to 
\citep{quinnEffectAnisotropicViscosity2020}, isotropic viscous heating
is found to be lower than anisotropic (switching)
viscous heating, by approximately two orders of magnitude. 

These numerical experiments have provided evidence that the flute instability can occur in twisted magnetic flux ropes and grow at similar rates to the kink instability. Further estimation of the relative growth rates in more realistic coronal loop setups is required to fully understand if the flute instability plays a pertinent role in coronal loop physics.

\section*{Acknowledgements}

We would like to thank David MacTaggart for his input on the thesis chapter on which this paper is based. Results were obtained using the ARCHIE-WeSt High Performance Computer
(\url{www.archie-west.ac.uk}) based at the University of
Strathclyde. JQ was funded via an EPSRC studentship: EPSRC DTG EP/N509668/1.

\bibliographystyle{mnras}
\bibliography{paper}

\appendix

\section{Associated software}

A custom version of Lare3d~\citep{arberStaggeredGridLagrangian2001} has been developed where a new module for anisotropic viscosity has been included. The version including the new module can be found at \url{https://github.com/jamiejquinn/Lare3d}, and has been archived at~\citep{keith_bennett_2020_4155546}. The version of Lare3d used in the production of the results presented here, including initial conditions, boundary conditions, control parameters and the anisotropic viscosity module, can be found in~\citet{keith_bennett_2020_4155625}. The data analysis and instructions for reproducing all results found in this report may be also found at \url{https://github.com/JamieJQuinn/coronal-fluting-instability-analysis} and has been archived~\citep{quinnJamieJQuinnCoronalflutinginstabilityanalysis2021}.

All simulations were performed on a single, multi-core machine with $40$ cores provided by Intel Xeon Gold 6138 Skylake processor running at $2$ GHz and $192$ GB of RAM, although this amount of RAM is much higher than was required; a conservative estimate of the memory used in the largest simulations is around $64$ GB. Most simulations completed in under $2$ days.

\bsp	% typesetting comment
\label{lastpage}
\end{document}
